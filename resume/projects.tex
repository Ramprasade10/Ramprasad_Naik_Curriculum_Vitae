%-------------------------------------------------------------------------------
%	SECTION TITLE
%-------------------------------------------------------------------------------
\cvsection{Projects}


%-------------------------------------------------------------------------------
%	CONTENT
%-------------------------------------------------------------------------------
\begin{cventries}


%---------------------------------------------------------
  \cventry
    {\href{}{}\quad \quad \href{}{}} % Role
    {BlockSQL} % Title
    {DBMS Project} % Location
    {Sep 2018 – Present} % Date(s)
    {
      \begin{cvitems} % Description(s)
        \item {The BlockSQL has been designed in collaboration with the college's Controller Of Examination Office to facilitate the Students, Teachers and Database Administrators of our college in the shortcomings like automating provisional marks card, transcript generation, previous results, and question papers. It is a secure and scalable application implementing the principles of blockchain into database to ensure data integrity, and consistency to help smoothen the operations at the COE’s office.
Data Analysis and visualization was used to generate useful graphs and metrics about student performance using matplotlib library from the provided data.
.\newline{}Technologies Used: Python(Flask,pdfkit,PyPDF2, passlib, pathlib, smtplib), SQLite3, matplotlib, tabula(Data Wrangling)}
      \end{cvitems}
    }

%---------------------------------------------------------


%---------------------------------------------------------
  \cventry
    {\href{https://blockhack.herokuapp.com}{blockhack.herokuapp.com}\quad\textbar\quad \href{https://github.com/Ramprasade10/BlockHack}{github.com/Ramprasade10/BlockHack}} % Role
    {BlockHack} % Title
    {TECHNIEKS HACKATHON} % Location
    {Feb 2018 – Present} % Date(s)
    {
      \begin{cvitems} % Description(s)
        \item {A Blockchain Ledger implementation using python would prevent hackers from rigging the electronic system to manipulate votes.Each vote(Block) is encrypted using sha256 hash,connected subsequently to other blocks, who’s integrity can never be compromised. This system enables a Secure, Reliable and let private individuals vote from home with biometrics, confirm their votes and who they voted.\newline{}
Technologies Used: Python flask(Server), HTML5, CSS,JSON.}
      \end{cvitems}
    }

%---------------------------------------------------------
  \cventry
    {\href{https://c-comp.herokuapp.com}{c-comp.herokuapp.com}\quad\textbar\quad \href{https://github.com/Ramprasade10/Online-C-Compiler}{github.com/Ramprasade10/Online-C-Compiler}}
    {Online C Compiler} % Title
    {NIE SUMMER OF CODE} % Location
    {June 2017 - Present} % Date(s)
    {
      \begin{cvitems} % Description(s)
        \item {Online-C-Compiler provides a clean User Interface for Compiling and Executing C Code Online without any compiler in the local system.Technologies used include flask(python),HTML5,CSS,AWS.The project aims to provide a platform for students to compile their code on the go, capabilities to share and commit code with ease.\newline{}
Technologies Used: Python, flask(Server), HTML5, CSS, AWS.}
      \end{cvitems}
    }

%---------------------------------------------------------
%---------------------------------------------------------
  \cventry
    {B GYAAN} % Role
    {Nourish} % Title
    {ONYX} % Location
    {Nov 17 - 18, 2017} % Date(s)
    {
      \begin{cvitems} % Description(s)
        \item {Presented a Business Plan before a panel of venture capitalist and entrepreneurs.
Nourish is an app/platform which aims to provide high quality nutritional food at affordable prices and cater to specific nutritional needs of fitness freaks.}
      \end{cvitems}
    }
%---------------------------------------------------------
%---------------------------------------------------------
  \cventry
    {\href{https://github.com/Ramprasade10/Cthru}{github.com/Ramprasade10/Cthru}} % Role
    {CThru} % Title
    {GTRAC} % Location
    {May 2015 – July 2015} % Date(s)
    {
      \begin{cvitems} % Description(s)
        \item {Android based App to help the blind see the world, a voice narrator guides the user of their surroundings.\newline{}
Technologies Used: REST API, Open Source library ocrapiservice, Java, Android.}
      \end{cvitems}
    }

%---------------------------------------------------------
%---------------------------------------------------------
  \cventry
    {\href{https://github.com/Ramprasade10/MedX}{github.com/Ramprasade10/MedX}} % Role
    {MedX} % Title
    {TECHNIEKS HACKATHON} % Location
    {Jan 2017} % Date(s)
    {
      \begin{cvitems} % Description(s)
        \item {MedX is an Android App that provides an all-round health care support by bridging the gap among patients, doctors and hospitals. A balanced ecosystem where patients find the best of doctors, and doctors pay a commission for each checkup. Features include in app symptom diagnostics, appointment and pill reminder.\newline{}
Technologies Used : Android, SQLite DB.}
      \end{cvitems}
    }

%---------------------------------------------------------
\end{cventries}
